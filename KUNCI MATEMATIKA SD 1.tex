\documentclass[12pt,a4paper]{article}
\usepackage[latin1]{inputenc}
\usepackage{amsmath}
\usepackage{amsfonts}
\usepackage{mathtools}
\usepackage{amssymb}
\usepackage{graphicx}
\usepackage[left=2.00cm, right=2.00cm, top=2.00cm, bottom=2.00cm]{geometry}
\author{H.O.W.K.E}
\title{KUNCI MATEMATIKA SD TAHAP I}
\begin{document}
	\maketitle
	\begin{enumerate}
		\item 1270
		\item 450
		\item n=4 berarti $\dfrac{2\cdot 4}{5}=\dfrac{8}{5}=1.6$
		\item FPB=5 KPK= 330
		\item KPK=24; 24 September 2018
		\item To:Ti=2:1 maka $\dfrac{To-Ti}{To+Ti}\times 42= \dfrac{1}{3} \times 42 = 14$ Tahun
		\item $\dfrac{B-A}{A+B}\times 30= \dfrac{1}{10}\times 30=3$
		\item 19441
		\item $\dfrac{16}{7}$
		\item 3
		\item $\dfrac{100}{20}\cdot 42=260$
		\item $K=\dfrac{J}{W}=\dfrac{0,5}{1,5}=\dfrac{1}{3}$ km/jam atau 333,4 m/jam
		\item 12.00
		\item 1007 hm atau yang sama.
		\item a. $\dfrac{1}{6},20\%,33\dfrac{1}{3}\%,\dfrac{1}{3},\dfrac{4}{5}$ atau dapat bertukar antara $33\dfrac{1}{3}\%$ dan $\dfrac{1}{3}$\\ b.$\dfrac{1}{4},\dfrac{12}{24},66\dfrac{2}{3},\dfrac{4}{3},1\dfrac{3}{4}\%$  
		\item Petunjuk : Mencari terlebih dahulu panjang sebenarnya dari skala. $L=(5\times500)x(2\times500)=2500000 cm^2=250 cm^2$
		\item 1 jam 14 menit
		\item 17 km
		\item Mencari volume= $10\times 8 \times 5=400m^3=400000L$, Menghitung debit= $\dfrac{400000}{100}=4000$ detik. $\dfrac{4000}{60}=66\dfrac{2}{3}$ menit. atau 1 jam 6 menit 40 detik.
		\item $K=4S$\\
			  $40=4S$\\
			  $S=10cm$\\
			  Jadi luasnya adalah $S^2=10^2=100cm^2$
		\item $L=\pi r^2$\\
			  $154=\dfrac{22}{7}r^2$\\
			  $154\cdot\dfrac{7}{22}=r^2$\\
			  $49=r^2$\\
			  $r=7$\\
			  jadi karena sisi persegi tersebut adalah diameter lingkaran maka sisi persegi tersebut adalah $r\times 2= 7 \times 2=14cm$
			  Jadi luas persegi tersebut adalah $s^2=14^2=196cm^2$ 
		\item $L=\dfrac{a+b}{2}\cdot t=((10+5)/2) \times 4= 30cm^2$
		\item diameter 10 cm maka radius adalah 5 cm. Dengan tinggi tabung 20 cm, maka $V=\pi r^2t=5^2\cdot 3.14\cdot 20= 1570cm^3$
		\item $L_t=6L_s$\\
			  $36=6L_s$\\
			  $6cm^2$\\
			  jadi panjang rusuk $r=\sqrt{6}$, maka volume $V=((\sqrt{6}))^3=6\sqrt{6}cm^3$\\
			  * soal ini adalah soal bonus
		\item Median = 19
	
	
			  
			  
	\end{enumerate}
		\paragraph{BAGIAN II}
	\begin{enumerate}
		\item Sudah Jelas
		\item Sudah Jelas
		\item Silahkan buat diagram lingkaran dengan ukuran derajat\\ Cemani=$60\textdegree$,Alas=$70\textdegree$,Kapas=$10\textdegree$,Pedaging=$100\textdegree$,Petelur=$120\textdegree$
		\item $T_1-T_0=(31\times 71)-(30\times 70)=225$, Nilai anak yang terakhir masuk adalah 225.
		\item $L=2(pl+pt+lt)=2(10+20+30)=120cm^2$
	\end{enumerate}	
\end{document}