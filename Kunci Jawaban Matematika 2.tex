\documentclass[12pt,a4paper]{article}
\usepackage[latin1]{inputenc}
\usepackage{amsmath}
\usepackage{amsfonts}
\usepackage{amssymb}
\usepackage{graphicx}
\usepackage[left=3.00cm, right=3.00cm, top=3.00cm, bottom=3.00cm]{geometry}
\author{H.O.W.K.E}
\title{KUNCI JAWABAN SOAL MATEMATIKA SMP II}
\begin{document}
	\maketitle
	\paragraph{BANGUN RUANG}
	\begin{enumerate}
		\item $V_t=\pi r^2 t=15\pi \cdot 10 = 150 \pi $ $cm^3$
		\item Jika diketahui Luas alas adalah $25\pi cm^2$, dapat kita cari radius dari  $L_\circ=\pi r^2=25\pi$ jadi $r=5$,  $L_T=\pi r(r+2t)=\pi = \pi 5(5+2 \cdot 7)=\pi 5(19)=95\pi$ $cm^2$
		\item $L_T=2\pi r(r+t)=2\pi 6(6+10)=12\pi(16)=192\pi cm^2$
		\item $V_{Kr}=\dfrac{1}{3}\pi r t = \dfrac{1}{3}\cdot 30\pi \cdot 3=30\pi cm^3$
		\item Untuk mencari luas sisi kerucut maka 
		tinggi sisi kerucut maka\\
		\\
		a. Cari Tinggi sisi kerucut "S"\\
		\\
		$S^2=r^2+t^2$
		\\
		$S^2=10^2+24^2$
		\\
		$S^2=100+576$
		\\
		$S^2=676$
		\\
		$S=\sqrt{676}$
		\\
		$S=26$ cm\\
		\\
		b. Cari luas sisi kerucut \\
		
		$L_k=\pi r(r+s)=10\pi(10+26)=10\pi(36)=360\pi$cm
		 
	\end{enumerate}

	\paragraph{PENYEDERHANAAN AKAR}
	\begin{enumerate}
		\item $\sqrt{212}=\sqrt{53} \cdot \sqrt{4}=2\sqrt{53}$
		\item $\sqrt{105}=\sqrt{105}$
		\item $\sqrt{45}=\sqrt{9}\cdot \sqrt{5}=3\sqrt{5}$
		\item $\sqrt{92}=\sqrt{4}\times \sqrt{23}=2\sqrt{23}$
		\item $\sqrt{72}=\sqrt{4}\times \sqrt{2}\times \sqrt{9}= 2\cdot 3\cdot \sqrt{2}=6\sqrt{2}$
	\end{enumerate}
	
	\paragraph{Penyederhanaan Operasi Akar}
	\begin{enumerate}
		\item Tidak bisa disederhanakan
		\item $4\sqrt{5}+11\sqrt{7}$
		\item $12\sqrt{10}-6\sqrt{13}$
		\item $\sqrt{75}+\sqrt{125}+\sqrt{250}=\sqrt{3 \cdot 25} + \sqrt{5\cdot 25} + \sqrt{10\cdot 25}= 5\sqrt{3} + 5 \sqrt{5}+ 5\sqrt{10} $
		\item $\sqrt{238}+14\sqrt{2}$
	\end{enumerate}
	\paragraph{Pythagoras}
	\begin{enumerate}
		\item Dapat digambar sendiri dengan siku-siku di $\measuredangle A$ dan hipotenusa 5, sisi terpendek 3 dan sisi panjang 4.
		\item BONUS, seharusnya hipotenusa $\overline{RQ}$ cm jadi sisi alas adalah $5\sqrt{3}$ cm dan sisi tinggi adalah 10 cm
		\item Karena segitiga istimewa sama kaki dengan kaki masing-masing 3 cm maka sisi yang lain adalah $3\sqrt{2}$. Maka sisi ini dapat dijadikan alas. 
		\\
		Kita dapat membuat garis tinggi dengan cara menarik garis tegak lurus ke $\overline{YZ}$, semisal $\overline{XG}$
		maka akan terbentuk segitiga $\triangle YXG$ dan $\triangle XGZ$, 
		Sehingga panjang garis $\overline{XG}$ dapat, dihitung dengan menggunakan pythagoras, \\
		\\
		$\overline{XG}^2=\overline{YX}^2 - \overline{YG}^2 $\\
		$\overline{XG}^2 = 3^2 - (\dfrac{3}{2}\sqrt{2})^2 $
		$\overline{XG}=\sqrt{9-(9/4\cdot 2)}=\sqrt{\dfrac{18}{2}-\dfrac{9}{2}}=\sqrt{\dfrac{9}{2}}=\dfrac{\sqrt{9}}{\sqrt{2}}=\dfrac{3}{\sqrt{2}}=\dfrac{3}{2}\sqrt{2}$ cm
		\\
		\\
		Perhatikan gambar dibawah: 
		\\
		\begin{center}
		\includegraphics[scale=0.5]{../Pictures/HOWKE/g4205}
		\end{center}
		
		Sehingga dapat dihitung luas segitiganya dengan menggunakan rumus luas segitiga yaitu:\\
		
		$L_\triangle = \dfrac{1}{2}\cdot a \cdot t= \dfrac{1}{2}\cdot 3\sqrt{2}\cdot \dfrac{3}{2}\sqrt{2}=\dfrac{9}{2}=4.5cm^2$
		\item Penjelasan :
		\begin{enumerate}
			\item a=$\sqrt{9^2-12^2}=\sqrt{81+144}=\sqrt{225}=15$
			\item b=$\sqrt{7^2+13^2}=\sqrt{49+169}=\sqrt{218} $
			\item 
			segitiga istimewa maka c=$4.5\sqrt{2}$
			\item Segitiga istimewa maka d=10
		\end{enumerate}
		\item Waktu keberangkatan Mayang dan Mira adalah 10.30 dan berakhir di pulau G dan T bersamaan pada pukul 12.00, jadi lama waktu keberangkatan adalah 1,5 jam.Jika kecepatan mayang adalah 12km/jam dan Mira 24 km/jam maka jarak mereka 1,5 jam kemudian adalah 
		
		$J_{Mayang}=K.W=1.5 \times 12= 18 km$\\
		\\
		dan,\\
		$J_{Mira}=K.W=1.5 \times 24= 36 km$
		
		Maka jarak pulau G dan T dapat dihitung dengan pythagoras,
		
		$b^2=36^2-18^2$\\
		$b^2=972$\\
		$b=\sqrt{972}=18\sqrt{3}$ km
		
		\item Penjelasan:
		\begin{enumerate}
			\item $e^2=12^2+7^2=95$\\
				  $e=\sqrt{95}$
			\item f=60 (sudah jelas)
			\item mencari tinggi segitiga, $x^2=8^2-(4+2)^2=100$ jadi $x=\sqrt{100}=10$. Kemudian g dapat dicari dengan $g^2=10^2+4^2=116$ jadi $g=\sqrt{116}=2\sqrt{29}$ 
		\end{enumerate}
	\end{enumerate}
		\paragraph{Kesebangunan dan Kongruensi}
	\begin{enumerate}
		\item Penjeleasan:
		\begin{enumerate}
			\item $\overline{AD}^2=16 \cdot 10=160$\\	
			$AD^2=\sqrt{160}=4\sqrt{10}$ cm
			\item $\overline{AB}^2=\overline{AD}^2+ \overline{BD}^2=(4\sqrt{10})^2 + 10^2=260$ jadi $\overline{AD}=\sqrt{260}=2\sqrt{65}$ cm
			\item $\overline{AC}^2=\overline{AB}^2+ \overline{BC}^2=\sqrt{260}^2+26^2=260+676=936$ jadi $AC=\sqrt{936}=6\sqrt{26}$ cm
			\item $\overline{BC}=26$ cm
			\item Luas $\triangle ABC=\dfrac{1}{2}\cdot \overline{AB} \cdot \overline{AC}=\dfrac{1}{2}\times 2\sqrt{65} \times 6\sqrt{26}= 76\sqrt{10}$ $cm^2$
			\item Luas $\triangle ADB=\dfrac{1}{2}\cdot \overline{BD} \cdot \overline{AD}=\dfrac{1}{2}\times 10 \times 4\sqrt{10}= 20\sqrt{10}$ $cm^2$
		\end{enumerate}
		\item $\dfrac{QU}{UR}=\dfrac{PQ}{TU}$\\
		\\
		$\dfrac{6}{8}=\dfrac{9}{TU}$
		\\
		$6TU=72$\\
		\\
		$TU=12$ cm\\
		\\
		dan \\
		\\
		$\dfrac{TU}{SR}=\dfrac{QU}{UR}$\\
		\\
		$\dfrac{12}{SR}=\dfrac{6}{8}$\\
		\\
		$6SR=96$\\
		\\
		$SR=\dfrac{96}{6}=16$ cm
		
		\item $\dfrac{30}{30-2-2}=\dfrac{20}{20-2-x}$\\
		\\
		 Foto dianggap \textit{portrait}. \\ 
		 \\
		$\dfrac{30}{26}=\dfrac{20}{18-x}$\\
		\\
		$30(18-x)=26\cdot 20$\\
		\\
		$540-30x=520$\\
		\\
		$-30x=-20$\\
		\\
		$x=\dfrac{2}{3}$ cm
		\item $\dfrac{DE}{BC}=\dfrac{AE}{AC}$\\
		\\
		$\dfrac{3}{x}=\dfrac{16-x}{16}$\\
		\\
		$16x-x^2=48$\\
		\\
		$x^2-16x+48=0$\\
		\\
		$(x-12)(x-4)$\\
		\\
		jadi x memiliki solusi 12 dan 4
		\item Dapat dibuktikan dengan metode SAS
	\end{enumerate}
	
\end{document}