\documentclass[12pt,a4paper]{article}
\usepackage[latin1]{inputenc}
\usepackage{amsmath}
\usepackage{amsfonts}
\usepackage{amssymb}
\usepackage{graphicx}
\usepackage[left=3.00cm, right=3.00cm, top=3.00cm, bottom=3.00cm]{geometry}
\author{Balya Rochmadi}
\title{PEMBUKTIAN BAHWA KONGRUENSI DUA SEGITIGA TIDAK DAPAT DILAKUKAN DENGAN METODE "SSA (Sisi, Sisi, Sudut)"}
\begin{document}
	\maketitle
	\paragraph*{}
	Yth. Bapak Sumantri, SP.d, MP.d.
	
	Pembuktian bahwa dua segitiga itu sebangun dan kongruen tidak bisa bisa menggunakan metode Sisi,Sisi, Sudut. Berikut ini saya memberikan \textit{counterexample} : ,
	
	
	
	\paragraph*{}
		Perhatikan gambar berikut!
	\begin{center}
	\includegraphics[scale=0.7]{../Pictures/HOWKE/g11809}
	\end{center}

	\begin{center}
	\includegraphics[scale=0.7]{../Pictures/HOWKE/g11827}
	\end{center}
	
	Secara jelas dapat dilihat bahwa jika dibuktikan sisi $\overline{PQ} \cong \overline{AB}$ dan $\overline{PR} \cong \overline{AB}$ dengan sudut yang sama yaitu $\measuredangle R \cong \measuredangle C$. Jelas bahwa dengan metode SSA alias jika terdapat dua sisi yang sama dan diikuti oleh sudut yang sama pada kedua segitiga. Maka dapt disimpulkan bahwa $\triangle PQR \cong \triangle ABC$. Tapi lihatlah dengan kasat mata bahwa $\triangle PQR$ dan $\triangle ABC$ \textbf{TIDAK SAMA DAN SEBAGUN BENTUKNYA!}, kenapa? Karena terdapat sudut bebas yaitu $\measuredangle P$ dan $\measuredangle A $ alias\textbf{ KEDUA SEGITIGA TERSEBUT TIDAK KONGRUEN!}
	
	Berarti anda salah, \textbf{mohon segera dibetulkan}.\\
	
	Salam,
	Balya Rochmadi, S.E, B.Eng, B.A, M.Eng, M.A.\\
	Alumni Universitas Gadjah Mada, Fakultas Ekonomika dan Bisnis.\\
	Alumni \textit{Atlanta University, Computer Science Faculty},\\
	Alumni \textit{Islamic Online University Cairo/ Al-Azhar}\\
	
	
	
	
	
\end{document}